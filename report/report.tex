\documentclass[pageno]{jpaper}

%replace XXX with the submission number you are given from the ISCA submission site.
\newcommand{\IWreport}{2015}
\newcommand{\step}{\longrightarrow}
\newcommand{\bstep}{\Downarrow}

\usepackage[normalem]{ulem}
\usepackage{amsfonts}
\usepackage{amssymb,amsmath}
\usepackage{bussproofs}
\usepackage{syntax}
\usepackage{textgreek}
\usepackage{listings}
%\usepackage{easy-todo}
\usepackage{color}
\usepackage{placeins}
\usepackage{stackengine}


\begin{document}

\definecolor{mauve}{rgb}{0.88, 0.69, 1.0}
\lstset{ %
  backgroundcolor=\color{white},   % choose the background color; you must add \usepackage{color} or \usepackage{xcolor}
  basicstyle=\ttfamily,        % the size of the fonts that are used for the code
  breakatwhitespace=false,         % sets if automatic breaks should only happen at whitespace
  breaklines=true,                 % sets automatic line breaking
  captionpos=b,                    % sets the caption-position to bottom
  %commentstyle=\color{green},    % comment style
  commentstyle=\ttfamily,
  deletekeywords={...},            % if you want to delete keywords from the given language
  escapeinside={\%*}{*)},          % if you want to add LaTeX within your code
  extendedchars=true,              % lets you use non-ASCII characters; for 8-bits encodings only, does not work with UTF-8
%  frame=single,                    % adds a frame around the code
  keepspaces=true,                 % keeps spaces in text, useful for keeping indentation of code (possibly needs columns=flexible)
  keywordstyle=\color{blue},       % keyword style
  language=Caml,                  % the language of the code
  otherkeywords={*,...},            % if you want to add more keywords to the set
  numbers=none,                    % where to put the line-numbers; possible values are (none, left, right)
  numbersep=5pt,                   % how far the line-numbers are from the code
  numberstyle=\tiny\color{gray}, % the style that is used for the line-numbers
  rulecolor=\color{black},         % if not set, the frame-color may be changed on line-breaks within not-black text (e.g. comments (green here))
  showspaces=false,                % show spaces everywhere adding particular underscores; it overrides 'showstringspaces'
  showstringspaces=false,          % underline spaces within strings only
  showtabs=false,                  % show tabs within strings adding particular underscores
  stepnumber=2,                    % the step between two line-numbers. If it's 1, each line will be numbered
  stringstyle=\color{mauve},     % string literal style
  tabsize=2,                       % sets default tabsize to 2 spaces
  title=\lstname                   % show the filename of files included with \lstinputlisting; also try caption instead of title
}

\newcommand{\lsti}{\lstinline}
\newcommand{\lstiH}{\lsti[language=Haskell]}

\title{Typsy: An ML-like programming language with typeclasses}

\date{}
\maketitle

\thispagestyle{empty}

\begin{abstract}

\end{abstract}

\section{Introduction}

In this paper, I describe the development of a programming language with a sophisticated type system.
Starting from the untyped lambda calculus, I add simple types, parametric polymorphism, type operators,
and finally the Typerec construct. These features allow us to encode Haskell typeclasses in an elegant way.


\section{Untyped Lambda Calculus}
While the Turing machine is the model of computation that gets the most popular attention,
and intuitively corresponds to the low-level behavior of computers, it is messy
and difficult to reason about.

The most natural and elegant way to model high level programming languages is
with various kinds of of lambda calculi, which are a collection of logical systems
for expressing and reasoning about functions.  Typsy is one
such system, albeit embellished with several features.

I'll start by discussing the simplest lambda calculus---the untyped one---on
which all other versions are built.

In the untyped calculus, all we have are two operations: \textbf{function abstraction},
and \textbf{function application}.
We also assume that we have an infinite set of variables.

In OCaml, we can represent expressions in this language with the type:
\begin{lstlisting}

type variable = string
type exp = 
   | Var of variable
   | Fun of variable * exp
   | App of exp * exp

\end{lstlisting}

The $\lambda$ expression represents function abstraction. The abstraction $\lambda x.t$ denotes
a function taking an argument $x$ and returning $t$. The term consisting of two adjacent sub-terms $t_1 t_2$ represents the application of $t_1$ to $t_2$.

Since these terms represent computer programs that are actually supposed to \textit{do} something, we want to
define a relation that says what a given expression evaluates to.

The real work of evaluation happens in function application. What does $(\lambda x.t_1)t_2$ evaluate to?
It seems clear that we want to replace occurrences $x$ in $t_1$ with $t_2$.
For we want these $\lambda$ terms to behave like ordinary functions, and this is how ordinary functions work.
For instance, if $f(x) = x + 1$, $f(13) = 13 + 1 = 14$, where we've applied $f$ to $13$ by substituting
$13$ for all the instances of $x$ in the definition of $f$.

But substituting for \textit{all} the occurrences of the variable turns out to be too much.
For instance, in the function application $(\lambda x. x (\lambda x. x y)) z$, we don't want to replace the inner
$x$ because it is bound in a different $\lambda$ than the one being applied.
So we want this expression to evaluate to $z (\lambda x. x y)$. The crucial distinction here is between
free and bound variables. Essentially, an occurrence of a variable $x$ is bound if there is an enclosing $\lambda$ abstraction taking $x$ as an argument.  It's free otherwise. So in the expression $\lambda x. x y$,
$x$ is bound and $y$ is free.  So, the proper way to substitute a term for a variable is to replace only
the free occurrences of the variable. 

There is one more kink we need to iron out in the definition of substitution. Consider the expression
$e = (\lambda x. \lambda y. x y) y$. If we substitute for all the free occurrences of $x$ as described, we get
$\lambda y.y y$. But now consider $e' = (\lambda x. \lambda z. x z) y$. I've only renamed $y$ to $z$, so
$e$ is in some sense equivalent to $e'$. But substituting y in $e'$ gives $\lambda z.y z$, which is clearly not equivalent to $\lambda y. y y$. If we want equivalent expressions to evaluate to equivalent results, we need to
change the way we perform substitution. The problem in $e$ is essentially that the $y$ starts out free, but is
\textit{captured} when we substitute it, becoming bound.  We want to avoid this, and we can do so by
renaming the bound variable to something other than $y$. (This is called $\alpha$-conversion.)

If we denote substitution of $x$ with $f$ in $e$ by $[x \mapsto f]e$,
and the free variables of $e$ by $FV(e)$, our substitution function is as follows:
\[
\begin{aligned}
& [x \mapsto e]x &= \qquad &e\\
& [x \mapsto e]y &= \qquad &y  && \text{if } y \neq x\\
& [x \mapsto e](\lambda y. f) &= \qquad &\lambda y. [x \mapsto e] f && \text{if } y \neq x \text{ and } y \notin FV(e) \\
& [x \mapsto e](e_1 e_2) &= \qquad &([x \mapsto e] e_1) ([x \mapsto e] e_2)
\end{aligned}
\]

Note that, in the case where we're substituting into a lambda expression, the substitution function
isn't total.  Specifically, it doesn't cover the cases where $x = y$, or where $y \in FV(e)$.
In these cases, we can $\alpha$-convert so that the necessary conditions hold.

Now that we've defined substitution precisely, we can use it to define 
an evaluation relation.\footnote{Actually, this evaluation relation doesn't need the above sophisticated substitution function,
because it uses a call-by-value evaluation order. But I will need an analogous substitution function
later for use with types.}
We say $e_1 \step e_2$ to mean that $e_1$ small-steps to $e_2$, i.e. that one
step of computation reduces $e_1$ to $e_2$. 

\begin{prooftree}
\AxiomC{$e_1 \step e_1'$}
\UnaryInfC{$e_1 e_2 \step e_1'e_2$}
\end{prooftree}

\begin{prooftree}
\AxiomC{value $v$}
\AxiomC{$e_2 \step e_2'$}
\BinaryInfC{$v \, e_2 \step v \, e_2'$}
\end{prooftree}

\begin{prooftree}
\AxiomC{value $v$}
\UnaryInfC{$(\lambda x.e_1)v \step [x \mapsto v]e_1$}
\end{prooftree}

This relation defines a call-by value evaluation order, where (1) functions and arguments are fully
evaluated before being applied, and (2) evaluation does not go inside the bodies of lambda expressions
(except by substitution).

Translating into code, we get:

\begin{lstlisting}

let eval e =
  match e with
  | Var v -> e
  | App (e1,e2) ->
     let e1 = eval e1 in
     let e2 = eval e2 in
     (match e1 with
      | Fun (v,e1') -> substitute e1' v e2
      | _ -> App (e1,e2))
  | Fun _ -> e

\end{lstlisting}

\subsection{Recursion}
Miraculously, these $\lambda$-expressions along with this evaluation function are a Turing-complete system---that is, any computable function can be written in it.
But any interesting computation involves looping or recursion, and it's not clear how we can do these
things in the lambda calculus---a system which lacks looping constructs and in which all functions are anonymous. It turns out that we can basically simulate recursion without any actual self-reference.
We just need something called a fixpoint-combinator.  I won't go into detail about how the fixpoint
combinator works.  The important point is that we don't need any extra constructs to implement recursion---a
property which won't hold for later systems.

\subsection{Environment-based Evaluation}
It is more efficient (and maybe more elegant) to be lazy in our performance of substitutions.
Rather than carry them out fully right away, we can keep track of them in an \textit{environment}
and apply them as needed.  Basically, an environment is a set of deferred substitutions, and it can
be represented by a mapping from variables to values (they will always be values,
since the application rule demands that the argument be fully evaluated already).
I use OCaml maps to represent environments.

\begin{lstlisting}
type env = exp Map.Make(String).t
\end{lstlisting}

There are a few changes that need to be made now:

\begin{itemize}
\item When applying a function, rather than performing a substitution, just bind a variable in
the environment.
\item When evaluating a variable, lookup it's value in the environment.
\item When evaluating a function, save the values of all of its free variables in a \textit{closure}.
Otherwise, when we use the function later in a (possibly) different environment, its free variables
may be shadowed.
\item The evaluation function now needs to be parameterized by an environment
\end{itemize}

%The other point is that it is impractical to implement a programming language by simply transcribing the
%small-step semantics into an interpreter. Such an implementation would have to traverse all the way down
%to the bottom of an expression for each step of evaluation, resulting in many traversals in total.
%There is another style of expressing operational semantics which is better suited to being transcribed
%into a working implementation: big-step semantics.  While the $\step$ relation tells us what an expression
%reduces to in a single computation step, the big-step $\bstep$ relation tells us what an expression
%reduces to when fully evaluated.

\section{Simple Types (STLC)}
Although this simple system is Turing-complete and we can encode numbers and booleans and operations on them in it, such encodings aren't particularly efficient.
So let's add some primitive constant expressions to our language, so that booleans and numbers
are represented directly as OCaml booleans and numbers. We then also need to add
some primitive operators so that we can manipulate these constant expressions.
So our new expression datatype is

\begin{lstlisting}

type variable = string

type constant = Int of int | Bool of bool
type operator = Plus | Times | Minus | ...

type exp = 
   | Var of variable
   | Fun of variable * exp
   | App of exp * exp
   
   | Constant of constant
   | Op of exp * operator * exp
   | If of exp * exp * exp

\end{lstlisting}

We then have to add some new evaluation rules for the new expressions.
So we evaluate arguments first:

\begin{prooftree}
\AxiomC{$e_1 \step e_1'$}
\UnaryInfC{$e_1 + e_2 \step e_1' + e_2$}
\end{prooftree}

\begin{prooftree}
\AxiomC{$e_2 \step e_2'$}
\UnaryInfC{$n + e_2 \step n + e_2'$}
\end{prooftree}

And then apply the operator:

\begin{prooftree}
\AxiomC{}
\UnaryInfC{$n_1 + n_2 \step n_1 \, +_{ML} \, n_2$}
\end{prooftree}

(I'm writing $+_{ML}$ for the \textit{actual} plus operator in the implementation language).
The other integer operators are analogous.

In the case of $if...then...else$ expressions, we \textit{do not} evaluate them in the
normal call-by-value manner, for we only ever want to evaluate one of the branches.

\begin{prooftree}
\AxiomC{$e_1 \step e_1'$}
\UnaryInfC{if $e_1$ then $e_2$ else $e_3 \step$ if $e_1'$ then $e_2$ else $e_3$}
\end{prooftree}

\begin{prooftree}
\AxiomC{}
\UnaryInfC{if $true$ then $e_1$ else $e_2 \step e_1$}
\end{prooftree}

\begin{prooftree}
\AxiomC{}
\UnaryInfC{if $false$ then $e_1$ else $e_2 \step e_2$}
\end{prooftree}

But by introducing operations that are restricted to one or the other primitive,
we create the possibility of expressions getting \textit{stuck} in evaluation.
That is, they might reach a state where (1) they aren't values and (2) they can't take a step. For instance the expression $true + 1$ is stuck. There's a sense in
which stuck expressions are meaningless, and we want a way to recognize statically which programs \textit{might} end up in stuck states.

\subsection{Typechecking}
This is where typechecking comes in.  We introduce a set of types and typing relation such that $e:T$ means that expression $e$ has type $T$. And we say $e$ is \textit{well-typed} if it has \textit{some} type. If we define the typing relation properly, an expression's well-typed-ness will guarantee that it won't get stuck in evaluation.

In addition to the base types $Int$ and $Bool$, we want functions to be well-typed, so we need function types. Our set of types could be represented as:

\begin{lstlisting}
type typ = 
  | IntT
  | BoolT
  | FunT of typ * typ
\end{lstlisting}

\subsubsection{Variables}
Since variables can be of any type, in order to (easily) typecheck expressions with variables in them, we need to annotate variables with a type when they are bound in $\lambda$ terms.  Furthermore, as we're typechecking, we need to carry this information around in a \textit{context}. Contexts are similar to environments; but instead of mapping variables to
values, contexts map variables to their types. For instance, in order to typecheck the expression $x + 1$, we need a context telling us the type of $x$. If $x$ was annotated with
type $Int \rightarrow Bool$, then the expression is \textit{not} well-typed, but if
it was annotated with type $Int$, then the expression is well-typed. So we should modify our
notation now, since expressions can only have type relative to a given context.
To say that $e$ has type $T$ in context $\Gamma$, we write $\Gamma \vdash e:T$.

\subsubsection{Typing Rules}
The typing rules are fairly straightforward:

\begin{prooftree}
\AxiomC{}
\UnaryInfC{$\Gamma \vdash n : Int$}
\end{prooftree}

\begin{prooftree}
\AxiomC{}
\UnaryInfC{$\Gamma \vdash true : Int$}
\end{prooftree}

\begin{prooftree}
\AxiomC{}
\UnaryInfC{$\Gamma \vdash false : Int$}
\end{prooftree}

\begin{prooftree}
\AxiomC{$x \in \Gamma$}
\UnaryInfC{$\Gamma \vdash x : T$}
\end{prooftree}

\begin{prooftree}
\AxiomC{$\Gamma \vdash e_1 : Int$}
\AxiomC{$\Gamma \vdash e_2 : Int$}
\BinaryInfC{$\Gamma \vdash e_1 + e_2 : Int$}
\end{prooftree}

\begin{prooftree}
\AxiomC{$\Gamma \vdash e_1 : Bool$}
\AxiomC{$\Gamma \vdash e_2 : T$}
\AxiomC{$\Gamma \vdash e_3 : T$}
\TrinaryInfC{$\Gamma \vdash$ if $e_1$ then $e_2$ else $e_3 : T$}
\end{prooftree}

\begin{prooftree}
\AxiomC{$\Gamma \cup \{x:T_1\} \vdash e:T_2$}
\UnaryInfC{$\Gamma \vdash (\lambda x:T_1.e) : T_1 \rightarrow T_2$}
\end{prooftree}

\begin{prooftree}
\AxiomC{$\Gamma \vdash e_1 : T_1 \rightarrow T_2$}
\AxiomC{$\Gamma \vdash e_2 : T_1$}
\BinaryInfC{$\Gamma \vdash e_1 e_2 : T_2$}
\end{prooftree}

\begin{itemize}
\item The first four rules are pretty self explanatory.
\item The rule for typing conditionals requires that both branches have the same type.
\item The rule for applications says that the function must accept arguments of the given type.
\item The rule for lambda abstractions says that a function maps from its argument type to its body type; and
as discussed above, since the body will likely use the function argument, we need to put it
in the context with the annotated type.
\end{itemize}

\subsubsection{Code}
These typing rules lend themselves to
easy translation into a working implementation. A context is just a map:
\begin{lstlisting}
module SM = Map.Make(String)
type context = typ SM.t
\end{lstlisting}

Since each constructor has exactly one corresponding rule, it follows by simple induction that
each expression has at most one type; meaning the typing relation is a (partial) typing function from expressions
to types. For a given rule, each premise just corresponds to a recursive call to this typing
function. And in the case where the same type occurs in two different premises of a rule (e.g. the application rule), this corresponds to a check that two types are equal. (At this point, two types are equal iff they are
syntactically equal, so the built in OCaml equality will do fine.)
Since the typing function is partial, we need to do something in the case that it's undefined on
an expression---meaning that the expression isn't well-typed.  I found it most convenient just to throw an
exception.

Here's the code:

\begin{lstlisting}
exception Type_error

let expect (t1:typ) (t2:typ) : unit =
  if t1 <> t2 then raise Type_error

let rec typeof (e:exp) (ctx:context) =
  match e with
  | Constant (Int _) -> IntT
  | Constant (Bool _) -> BoolT
  | If (cond,e1,e2) ->
      let tcond = typeof cond ctx in
      expect tcond BoolT ;
      let t1 = typeof e1 ctx in
      let t2 = typeof e2 ctx in
      expect e1 e2 ;
      e1
  | App(e1,e2) ->
      let t1 = typeof e1 ctx in
      let t2 = typeof e2 ctx in
      (match t1 with
       | FunT (t11,t12) -> expect t11 t2 ; t12
       | _ -> raise Type_error)
  | Fun (x,tx,body) ->
      let ctx' = SM.add x tx ctx in
      let tbody = typeof body ctx' in
      FunT (tx,tbody)
    
\end{lstlisting}
\subsection{Recursion}
However, STLC is \textit{not} Turing-complete---gasp!
We have no way of writing well-typed fixed-point combinators, \footnote{For instance, the Y-Combinator,
equal to $\lambda f. (\lambda x. f(x \,  x)) (\lambda x. f(x \, x))$, has a self-application $(x \, x)$
as a subterm.  But self-application is never well-typed. For if $x$ had a type, it would clearly
be a function type $T = T_1 \rightarrow T_2$. But by the application typing rule, $T$ must be equal
to $T_1$. So $T$ must contain itself as a subexpression, which is impossible.} and hence no way of doing general recursion.

So to endow our language with Turing-completeness once again, we have to add a new primitive.
In my implementation, there is a \lsti{Rec} constructor which is basically a named $\lambda$.
When a recursive function is called, we evaluate the body just like in the case of a non-recursive function,
except we also bind the name of the function to the function closure itself.

The evaluation rule might look like this:

\begin{prooftree}
\AxiomC{}
\UnaryInfC{$(rec \ f x. e) v \step [x \mapsto v][f \mapsto rec \ f x. e]e$}
\end{prooftree}

\subsection{Other types}
Once we have this infrastructure set up, it's pretty easy to add new types, such as strings, pairs, and lists.
\begin{itemize}
\item Strings have a concatenation operator, written $++$, an indexing operator written \#\footnote{e.g. ``1234''\#3 gives the third
character, i.e. ``4''}, and a length function $strlen$.
\item Pairs have $fst$ and $snd$ operators for getting the first and second components respectively.
\item Lists are constructed with $nil$ and the $cons$ operator, and destructed with $match$ statements.
\item I add a placeholder type \lsti{NoneT}, which represents some unknown type that the user
neglected to annotate. Almost all functional
programming languages have type-inference, whereby the compiler can determine the types of terms that
are unannotated. So the idea is that, using type-inference, all the \lsti{NoneT} types will be filled
in with actual types. \footnote{I implemented type inference for STLC, but not for the full system. Currently,
I have a ``stub'' implementation of type-inference which, rather than actually inferring anything, just
validates that there are no \lsti{NoneT} annotations present and throws an exception otherwise. I wanted
to build my system in such a way that type-inference could be smoothly added to it later without having to
refactor too much.}
\end{itemize}

\section{Polymorphic Lambda Calculus}
This language is cool as a starting point, but its type system is still pretty basic.
For instance, there's no way to write a well-typed polymorphic map function. If we want to map over lists of
booleans we can write a function to do that, but we won't be able to call it on a list of integers,
even though the function doesn't actually do anything boolean-specific with the elements.

How can we implement parametric polymorphism? We need to extend our language with \textit{type functions}---functions that take types as arguments, and return expressions as results. 
We can think of the polymorphic map function as being one such type function. When applied to a type $T$, it
returns a map function specific to type $T$.
Type functions can be written in the form $\Lambda X.e$, where $X$ is the type argument,
and $e$ is an expression in which $X$ can occur free. So the polymorphic identity function would be
written $id = \Lambda X.\lambda x:X.x$.

It's conventional to apply type functions by putting the type argument in brackets.  So the identity function
instantiated to type $Int$ would be $(\Lambda X.\lambda x:X.x) [Int]$

So now our expression datatype looks like:

\begin{lstlisting}
type exp =
    ...
  | TFun of variable * exp
  | TApp of exp * typ
\end{lstlisting}

But what is the type of the polymorphic map function? Nothing in our current type system covers it, so we need to
extend our language with \textit{universal types}. These types are of the form $\forall X. T$, where
$X$ is a \textit{type variable}, and $T$ is a type, in which $X$ can occur free.  If $e$
has type $\forall X.T$, we can think of this as meaning that $e$ has type $T$ \textit{for all} possible
types $X$.

The typing rules for $\forall$ types are:

\begin{prooftree}
\AxiomC{$\Gamma \cup \{X\} \vdash e : T$}
\UnaryInfC{$\Gamma \vdash \Lambda X.e : \forall X.T$}
\end{prooftree}

\begin{prooftree}
\AxiomC{$\Gamma \vdash e : \forall X.S$}
\UnaryInfC{$\Gamma \vdash e [T] : [X \mapsto T]S$}
\end{prooftree}

Note the use of type substitution in the second rule. We can define type substitution in
the same way we defined term substitution before.

Evaluation isn't changed very significantly by the addition of polymorphism.
It would be perfectly effective to simply ignore all type-abstractions and type-applications,
although the result of evaluation in the end might have an incomplete type-annotation.

However, for the purpose of implementing more sophisticated type features later, we need to pass
types around at runtime. So that means parameterizing the evaluation function by a \textit{type environment},
represented again as an map.

\section{Interlude: Parsing}

Up until this point, the language was simple enough that it was possible to develop and test it  without a parser---i.e. by manually constructing abstract syntax trees. But to make testing---and ultimately, use---practicable, I needed to write a parser for generating AST's from actual code.

\subsection{Menhir}

The tools I chose to use
are ocamllex and menhir, which are lexer and parser generators, respectively.
The lexer generates a stream of tokens which is then parsed into an AST by the parser.

Using ocamllex is really straightforward. Most kinds of tokens can be recognized by simple regular
expressions.
Using menhir is somewhat more involved. Menhir and parser generators like it---yacc, ocamlyacc, bison, etc---take as input a description of a language in the form of a BNF grammar, which is basically a set of rules
for recognizing valid expressions.  In a BNF grammar, there's a set of terminal symbols, or tokens, which are the simple units
of the language.  For instance \lsti{->}, \lsti{x}, \lsti{fun}, and \lsti{+} are
examples of tokens (generated by the lexer). In Menhir, terminal symbols/tokens are declared with the \lsti{%token} keyword, and their names should match up with the tokens produced by the lexer.

The nonterminal symbols are the complex expressions built up
out of tokens.  In this case, the primary nonterminal symbols are \lsti{exp} and \lsti{typ}, in
addition to a number of auxiliary nonterminals. Each nonterminal symbol has a set of alternative ways it can be matched. For instance, the \lsti{exp} nonterminal can be matched by something of the form
\lsti{<exp> + <exp>}, or by something of the form \lsti{\ <id> => <exp>}, and so forth. The alternatives are separated by vertical bars. Thus, the syntax and semantics of BNF grammars are very much like
the syntax and semantics of algebraic datatypes.

Note also that, since we don't just want to \textit{recognize} a language---we also want to generate
an abstract syntax tree---we can put an OCaml expression in braces to the right of each alternative.

Given the analogy between grammars and datatypes, one might (and I did) naively write a grammar for the simply
typed lambda calculus as follows:

\FloatBarrier
\begin{figure}
\begin{lstlisting}

%token <int> INT
%token <string> STR
%token INTT
...
...

typ =
  | LPAREN; t = typ; RPAREN     { t }
  | INTT                        { IntT }
  | BOOLT                       { BoolT }
  | t1 = typ; ARROW; t2 = typ   { FunT(t1,t2) }

exp =
  | LPAREN; e = exp; RPAREN
  | FUN; v = ID; COLON; t = typ; BIGARROW; e = exp   { Fun (v,t,e) }
  | IF; e1 = exp; THEN; e2 = exp; ELSE; e3 = exp     { If (e1,e2,e3) }
  | e1 = exp; PLUS; e2 = exp                         { Op(e1,Plus,e2) }
  | e1 = exp; MINUS; e2 = exp                        { Op(e1,Minus,e2) }
  | e1 = exp; STAR; e2 = exp                         { Op(e1,Times,e2) }
  | e1 = exp; DIV; e2 = exp                          { Op(e1,Div,e2) }
  | e1 = exp; e2 = exp                               { App(e1,e2) }
  | n = INT                                    { Constant (Int n) }
  | TRUE                                       { Constant (Bool true) }
  | FALSE                                      { Constant (Bool false) }

\end{lstlisting}
\end{figure}
\FloatBarrier

This grammar is actually pretty close to being correct. But when menhir tries to compile it, it'll spit out
a scary number of ``shift-reduce'' errors, indicating grammar ambiguities.
There are a few kinds of ambiguities involved here
\begin{itemize}
\item Arithmetic expressions. For instance \lsti{3 + 4 * 7} is ambiguous in the current grammar
\item Function application. In \lsti{(\ (x:Int) => x) 5 + 3}, which has higher precedence, function
      application, or addition?
\item Other ambiguities.  For example, \lsti{\ (x:Int) => x + 1} could be parsed as
      \lsti{(\ (x:Int) => x) + 1} or as \lsti{\ (x:Int) => (x + 1)}.  Although it's obvious to us
      which of these is intended, this is only because of our mental typechecking.  The parser doesn't know
      about types, so as far as it knows, a function can be added to an integer.
\end{itemize}
The arithmetic ambiguities
can be resolved by specifying operator precedence and associativity, using \lsti{%prec}, \lsti{%left}, \lsti{%right}, and \lsti{%noassoc}. 

The other shift-reduce errors, as far as I know, cannot be fixed with precedence and associativity annotations
in menhir.  Instead, we have to add a few auxiliary nonterminal symbols.  We can do it like so:

\FloatBarrier
\begin{figure}
\begin{lstlisting}

exp =
  | FUN; v = ID; COLON; t = typ; BIGARROW; e = exp   { Fun (v,t,e) }
  | IF; e1 = exp; THEN; e2 = exp; ELSE; e3 = exp     { If (e1,e2,e3) }
  | e = exp1                                         { e }
 
exp1 = 
  | e1 = exp1; PLUS; e2 = exp1                       { Op(e1,Plus,e2) }
  | e1 = exp1; MINUS; e2 = exp1                      { Op(e1,Minus,e2) }
  | e1 = exp1; STAR; e2 = exp1                       { Op(e1,Times,e2) }
  | e1 = exp1; DIV; e2 = exp1                        { Op(e1,Div,e2) }
  | e = exp2                                         { e }

exp2 =
  | LPAREN; e = exp; RPAREN                    { e }
  | e1 = exp1; e2 = exp1                       { App(e1,e2) }
  | n = INT                                    { Constant (Int n) }
  | TRUE                                       { Constant (Bool true) }
  | FALSE                                      { Constant (Bool false) }

\end{lstlisting}
\end{figure}
\FloatBarrier

This creates a hierarchy of expressions.  We can basically think of \lsti{exp}'s, \lsti{exp1}'s, and \lsti{exp2}'s as having low, medium, and high precedence respectively.
This ranking happens because an \lsti{exp1} can't contain an \lsti{exp} as a subexpression without
parentheses. But an \lsti{exp} can contain an \lsti{exp1} as a subexpression because of the last
alternative in the \lsti{exp} rule.  The same relationship holds between \lsti{exp1}'s and
\lsti{exp2}'s.

Extending this parser to the full language is straightforward given this technique.


\subsection{Syntax Transformations}

It's convenient to be able to write multi-argument functions and let expressions
even though these constructs aren't actually present in the $\lambda$ calculus.
So what we can do is to have a surface language which has these features, and then compile
the surface language into the simpler one which is easier to typecheck and execute.
This is exactly what I do: the surface language is in the module
\lsti{Parsed_syntax} and the compilation into the target language happens in the \lsti{Convert} module.

The actual syntax transformations aren't terribly complicated.
There are two main operations that happen:
\begin{itemize}
\item Getting rid of \lsti{let} expressions, by turning them into function applications.
An expression like \lsti{let x:Int = 5 in x + 24} wil turn into \lsti{(\ (x:Int) => x + 24) 5}.
\item ``Currying'' multi-argument functions by turning them into nested, single-argument functions.
An expression like \lsti{\ (x:Int) (y:Int) (z:Int) => x + y + z} will be curried into
\lsti{\ (x:Int) => \ (y:Int) => \ (z:Int) => x + y + z}.
\item We can also convert more complex \lsti{let} expressions.
For instance, given the expression 

\begin{lstlisting}
  let f (x:Int) (y:Int) : Int = x + y in
  f 73 5 
\end{lstlisting}
 
we first notice that this is equivalent to

\begin{lstlisting}
  let f : Int -> Int -> Int =
      (\ (x:Int) (y:Int) => x + y)
  in
  f 73 5
\end{lstlisting}

(The type of \lsti{f} is always just the concatenation of the types of the arguments with the return type.)

Then we can curry the definition of \lsti{f}, and transform the \lsti{let} as above, resulting in
the final expression:

\begin{lstlisting}
  (\ (f:Int->Int->Int) => f 73 5)
    (\ (x:Int) (\ (y:Int) => x + y))
\end{lstlisting}
\end{itemize}

There's one slight inconvenience of doing this transformation on \lsti{let} expressions, however.
Since this language doesn't (yet) have type inference, we need to annotate all function parameters
with their types. And since \lsti{f} in this example gets transformed into a function parameter,
it too must be annotated with a type, even though its type is just equal to the type of its definition.
Possible solutions to this issue are:

\begin{itemize}
\item Add type inference to the language. With type inference, the type of a function
can be determined from the context in which it's used. So in the expression,

\begin{lstlisting}
  (\ f => f 73 5)
    (\ (x:Int) (\ (y:Int) => x + y))
\end{lstlisting}

the type inference algorithm would figure out that \lsti{f} must have type \lsti{Int->Int->Int} because
\lsti{(\ (x:Int) (\ (y:Int) => x + y))} has type \lsti{Int->Int->Int}.
\item Simply allow \lsti{let} expressions in the target language.  Then the type of \lsti{f} could
just be determined from the type of its definition. I chose not to do this because I want to keep the
target language as simple as possible, and because type-inference is a something I'd like to try long-term
anyway.

%\item Run the typechecker during the syntax transformation process. So, given an unannotated \lsti{let}
%like \lsti{let x = 5 in x + 24}, we'd first get the type of the definition of \lsti{x} (namely \lsti{Int})
%before desugaring the \lsti{let} into \lsti{(\ (x:Int) => x + 24) 5}. This would actually be a cumbersome
%change to make, however: I'd have to keep track of 
\end{itemize}

\subsection{Notes on syntax}

At first I designed the syntax such that every function construct---non-recursive value-level functions, recursive value-level functions, non-recursive type-abstractions, and (as I'll get to shortly) recursive type-abstractions---was written differently. There were \lsti{fun}'s, \lsti{tfun}'s, \lsti{rec}'s, and
\lsti{trec}'s. This was unambiguous but cumbersome. To improve code clarity, I decided to
designate normal functions and type abstractions with the same backslash syntax.
The type of the function could be inferred by parameter names: type variables always start with an
uppercase letter, and regular variables, a lowercase letter. Finally, the \lsti{rec} keyword can be
prepended to any function uniformly.

Furthermore, type parameters and regular parameters can be mixed in a single function expression,
or a single let expression.

These changes \textit{vastly} improved the syntax.

\section{typecase, Typecase, and Typeclass}

\subsection{Typeclasses}
Typeclasses are a much-loved feature of Haskell, absent from the ML family of languages. They're basically
a mechanism for doing ad-hoc polymorphism, i.e. function overloading, with similar functionality to
Java interfaces. A typeclass declares a set of functions and their types.
For instance, in Haskell, the \lstiH{Eq} typeclass for equality can be defined as\footnote{The actual
definition of \lstiH{Eq} in the Haskell Prelude provides default implementations of \lstiH{/=} in terms of \lstiH{==}, and vice versa.}:

\begin{lstlisting}[language=Haskell]
class Eq a where
(==), (/=) :: a -> a -> Bool
\end{lstlisting}

Typeclasses are instantiated with \lstiH{instance} declarations, which must define the typeclass's functions for a specific type.  Each type can only have one instance defined for it. So the
instance declaration for lists might look like:

\begin{lstlisting}[language=Haskell]
instance Eq a => Eq [a] where
xs == ys = all (\ (x,y) -> x == y) (zip xs ys)
\end{lstlisting}

The \lstiH{=>} can sort of be read as a logical implication. It means that \textit{if} \lstiH{Eq a} is defined, \textit{then} \lstiH{Eq [a]} will be defined.
Then in our code, we can call \lstiH{==} on the elements of the given lists.

We can also define equality on trees like so:

\begin{lstlisting}[language=Haskell]
data Tree a =  Leaf | Node a (Tree a) (Tree a)

instance Eq a => Eq (Tree a) where
t1 == t2 =
  case t1,t2 of
  | Leaf,Leaf -> True
  | Node a1 l1 r1, Node a2 l2 r2 -> a1 == a2 && l1 == l2 && r1 == r2
  | _ -> False
\end{lstlisting}

And then, in our code, the \lstiH{==} function can be used identically on trees and lists.

It is true that we can do most of this with OCaml module types/signatures and modules/structures. Module types sort of define interfaces to be implemented by modules. So we could define an equality signature:

\begin{lstlisting}
module type Eq = 
sig
  type t
  val (==) : t -> t -> bool
end
\end{lstlisting}

And then we can implement this signature in two different ways:

\begin{lstlisting}
module ListEq(ElemEq:Eq) : (Eq with t = ElemEq.t list)
struct
  type t = ElemEq.t list
  let (==) l1 l2 =
    List.for_all2 (\ x1 x2 -> ElemEq.(==) x1 x2) l1 l2
end

type 'a tree = Leaf | Node of 'a * ('a tree) * ('a tree)
module TreeEq(ElemEq:Eq) : (Eq with t = ElemEq.t tree)
struct
  type t = ElemEq.t tree
  let rec (==) t1 t2 =
    match t1,t2 with
    | Leaf,Leaf -> true
    | Node (x1,l1,r1), Node (x2,l2,r2) ->
        ElemEq.(==) x1 x2 && l1 == l2 && r1 == r2
    | _ -> false
end

\end{lstlisting}

However, there is a significant advantage of typeclasses over modules in terms of convenience/brevity.
In both definitions here, to call the equality function on the elements of the trees/lists,
we have to call \lsti{ElemEq.(==)}. And we can't just open the \lsti{ElemEq} module, because
\lsti{ElemEq.(==)} would then clash with the \lsti{==} we're trying to define.

Similarly, when using our newly defined eqality functions, we can't mix calls to \lsti{tree} equality
and \lsti{list} equality, again because of name clashing. In fact, we can't even mix calls to
\lsti{int list} equality and \lsti{bool list} equality, because these two operations are defined in
separate modules! Compare this to the Haskell code above where we freely mix equality on lists with
equality on list elements.

This may seem like only a minor advantage, but when typeclasses are heavily used, as they are in Haskell, it adds up. Monadic-style programming would be impossible if one could only use one monad at a time.

Of course, the price of this brevity, which I mentioned above, is that you can only instantiate a typeclass
once per type.

\subsection{Overall Approach}
Now that it's clear what typeclasses are and why they're useful, I can describe the journey towards enriching System F with typeclasses (or something like them). There are two main properties
of typeclasses that we need to capture. First is ad-hoc polymorphism---that is, the ability to execute \textit{different code} depending on the type of an argument (or depending on some type argument).
The second is partiality: the ability to define a function on some limited set of types, such that if it's called on a type outside of this set, \textit{it won't typecheck}.

Typeclasses in GHC/Haskell are implemented with ``dictionary-passing'', but I'll take the approach described
in Harper and Morrisett \cite{harper-typecase} instead. This will mean extending the language in a few ways.  First, I'll add the
\lsti{typecase} construct, which allows us to easily implement ad-hoc polymorphism. Then I'll add
\textit{type-operators}, and the \lsti{Typecase} (with a capital 'T') construct, which will give us
partiality.

\subsection{typecase}
The \lsti{typecase} construct is the most natural way to execute different code depending on a
given type. It is analogous to a \lsti{match} statement but for types rather than terms.

The easiest way to explain how it works is by example. Here is a (pretty useless) string conversion function:

\begin{lstlisting}[mathescape]
let to_string A =
  typecase [T.T -> Str] A of
  | Int => int2string
  | Bool => bool2string
  | Str => id [str]
  | T1 -> T2 => (\ (f:T1 -> T2) => "function" )
  | T1 * T2 => (\ (p:T1*T2) => "pair")
  | List T => (\ (l:List T) => "list")
  end
in
...
\end{lstlisting}

When \lsti{to_string} is supplied with a type \lsti{A}, the \lsti{typecase} will match \lsti{A}
with one of its branches, and the whole expression will reduce to the value of that matched branch.
So in this case, the expression will reduce to a \lsti{to_string} function specialized to
type \lsti{A}.

Note that we need to bind type variables in evaluating the branches for the complex types. For example,
in evaluating \lsti{to_string [Int -> Bool]}, we need to bind \lsti{T1} to \lsti{Int} and \lsti{T2} to \lsti{Bool} before we evaluate the branch body. 
Given this observation, we can make the code a little bit more elegant
by representing these branches as type functions. So we can represent the function branch by
\lsti{(\ T1 T2 => <branch-body>)}, and likewise for the list and pair branches. This transformation
occurs in the \lsti{Convert} module.

The annotation \lsti{[T.T -> Str]} basically tells the typechecker that, when applied to type \lsti{[T]}, the \lsti{typecase} expression will reduce to something of type \lsti{T -> Str}. 
So the body of the \lsti{Int} branch must have type \lsti[mathescape]{[T $\mapsto$ Int](T->Str) $=$ Int->Str}.
Similarly, the body of the bool branch must have type \lsti{[bool->str]}, and so on.

%\todo{maybe give formal typechecking rules and evaluation semantics}

Wouldn't it be cleaner if we were to push the argument to \lsti{to_string} outside the typecase?
Could we write the following?

\begin{lstlisting}[mathescape]
let to_string = 
\ A (x:A) =>
  typecase [T.T -> Str] A of
  | Int => int2string x
  | Bool => bool2string x
  | Str => x
  | T1 -> T2 => "function"
  | T1 * T2 => "pair"
  | List T => "list"
  end
in
...
\end{lstlisting}

Alas, this much nicer function wouldn't typecheck. Specifically the \lsti{Int}, \lsti{Bool},
and \lsti{Str} branches would fail, because they mention the variable \lsti{x}. As far as the typechecker is concerned, \lsti{x} has type \lsti{A}, which represents any type. Even inside the \lsti{Int} branch,
the typechecker has no value bound to the type variable \lsti{A}. These are the typing rules presented by
Harper and Morrisett anyway. I very well may modify my typechecker to make the second version legal,
which would involve substituting \lsti{Int} for \lsti{A} throughout the type context, when typechecking
the \lsti{Int} branch, and likewise for the other branches. But things become more hairy when
the argument to the \lsti{typecase} is not just a simple type variable, but is instead a complex
type. 
%Given an expression like \lsti[mathescape]{typecase $\alpha$ of ...}, where
%$\alpha$ is not just a variable, what do we do? I'm not sure it's actually feasible.

\subsection{Type Operators}

Suppose we're not happy just printing \lsti{<function>} when \lsti{to_string} is called on a function,
and we want the typechecker to consider such calls erroneous.  This is where partiality comes in.
Harper and Morrisett present an elegant method of achieving this. There are two steps before
we can implement their method.

First we need to make a big addition to the typesystem, namely type operators.
We already have functions taking types as arguments and returning expressions, but a type operator takes
a type as argument and \textit{also returns a type}.
Type operators are basically a copy of the lambda calculus one level up.  In the literature, type operators are usually written with the same $\lambda$ syntax as regular term-level functions, but to differentiate the two, I use a double-slash \lsti{\\} in my language syntax for type operators. As an example, we can define
a pairing type operator which takes any type and returns the type of pairs of that type:

\begin{lstlisting}
\\A => A * A
\end{lstlisting}

But note that, if we want to treat type operators as ``first-class citizens'', on a par with the types
we've seen thus far, we should allow type operators to be passed around as arguments and returned by other
type operators. For example here is an operator that applies its first-argument (itself a type-operator)
to its second and third aruguments:

\begin{lstlisting}
\\A B C => A B C
\end{lstlisting}

Since we've lifted function abstraction from the term-level to the type level, we also now need to lift
typechecking. For instance, we want to be able to detect that the following type is malformed:

\begin{lstlisting}
(\\A B C => A B C) [Int]
\end{lstlisting}
since \lsti{Int} is a type that cannot be applied to other types.
In the same way that we assign each expression a type, we assign each type a kind, and during the
typechecking phase, we also perform kind-checking. Types like \lsti{Int}, \lsti{Bool}, and
\lsti{Forall A. A -> A}, we call proper types, and we give them kind \lsti{*} (pronounced ``type'').
(Expressions (i.e. at the term-level) can only have proper types as their type, i.e. there's no
expression with type \lsti{\\A => A * A}.)

If \lsti{k1} and  \lsti{k2} are kinds, then we can form a kind \lsti{k1 -> k2}, which is the kind of
type operators taking arguments of kind \lsti{k1} and returning results of kind \lsti{k2}. So the pairing operator above has kind \lsti{* -> *}.

\subsubsection{Normalization}

With the introduction of type operators, we now have the task of figuring out if two type operators are
equivalent, for the purpose of typechecking. For instance, given the program
\begin{lstlisting}
let x:(\\(T::*) => Int) [Bool] = 5
in x + 1
\end{lstlisting}
we need to check that \lsti{x}'s type annotation is equivalent to the type of its definition. (It is.)
Up until now, we could just check for syntactic equality up to renaming of bound variables, but this won't
suffice anymore as the above example shows.

The basic method to check for type-equivalence between two types \lsti{T1} and \lsti{T2} is to normalize
both types, and then check for syntactic equality (up to $\alpha$-equivalence).
What algorithm shall we use to normalize? 
It is tempting to argue that we can just do repeated $\beta$-reductions until
it's no longer possible, as we do at the value-level. Specifically, it's tempting to argue that we don't
have to reduce underneath $\lambda$ type-abstractions. The argument would be that we're only ever checking
for equivalence between proper types, and therefore, if we have a type-operator, it will be \textit{applied to something}, and so we can $\beta$ reduce in this case, rather than reducing underneath the $\lambda$.
This is what we do at the value-level. If we have two expressions of type \lsti{Int}, and we want to check
if they're equivalent, we just $\beta$-reduce until they're both values, and then check syntactic equality.

But the problem is that we might not be dealing only with closed types. To give an example, first let
\lsti{const_int} be the operator $\lambda (X::*). Int$
\begin{lstlisting}
\\(X::*) => Int
\end{lstlisting}

which always returns \lsti{Int} when applied to anything.

Then let \lsti{const_int'} be the equivalent operator
\begin{lstlisting}
\\(X::*) =>
    ((\\(Y::*) => Y) [Int])
\end{lstlisting}

which does the same thing but has a superfluous identity operator in its body.
Then consider the program:
\begin{lstlisting}
\ (T::(*->*)->*) (f:T const_int -> Int) (x:T const_int') => f x
\end{lstlisting}

To check that the function application \lsti{f x} is ok, we need to check the equivalence of the types
\lsti{T const_int} and \lsti{T const_int'}, and since we don't know what \lsti{T} is, we can't do any
$\beta$-reductions on it. Instead, we have to check that \lsti{const_int} and \lsti{const_int'} are equivalent,
which involves \textit{going underneath the $\lambda$}.

%So we can normalize types like this:
%
%\begin{lstlisting}
%let rec normalize_type (t:typ) : typ =
%  match t with
%  | BoolT | IntT | StrT | VarT _  -> t
%  | FunT (t1,t2) -> FunT (normalize_type t1,
%                          normalize_type t2)
%  | PairT (t1,t2) -> PairT (normalize_type t1,
%                            normalize_type t2)
%  | ListT u -> ListT (normalize_type u)
%  | ForallT (v,k,t') -> ForallT (v,k,normalize_type t')
%  | TFunT (v,k,t') -> TFunT (v,k,normalize_type t')
%  | TAppT (t1,t2) ->
%    let t1' = normalize_type t1 in
%    let t2' = normalize_type t2 in
%    (match t1' with
%     | TFunT (v,k,t1'_body) ->
%       let subbed = sub_in_typ t1'_body v t2' in
%       normalize_type subbed
%     | TRecT (f,v,k1,k2,t1'_body) ->
%       let subbed = sub_in_typ t1'_body v t2' in
%       let subbed = sub_in_typ subbed f t1' in
%       normalize_type subbed
%     | _ -> TAppT (t1',t2'))
%\end{lstlisting}

\subsection{Typecase/Typerec}

Harper and Morrisett introduce both a value-level \lsti{typecase} form, and a type-level \lsti{Typecase} form.
The two are analogous in that they both match on a type argument. However, the latter returns a type rather
than a term. As a use case Harper and Morrisett give the example of an type operator which
gives a representation for an array that depends on the type of the array elements, for more efficient use
of memory\footnote{This is just pseudocode to motivate the idea. In Typsy, there has to be a branch
for each of \lsti{Int}, \lsti{bool}, \lsti{str}, \lsti{void}, functions, products, and lists, in that order. (I'll explain
void soon).}:

\begin{lstlisting}[mathescape]
Typecase $\alpha$ of
| Bool => Boolarray
| Int => Intarray
| $\tau$ => Boxedarray ($\tau$)                    
\end{lstlisting}

More interestingly, there are cases in which we might want to be able to do recursion on the type argument.
For instance we might want to represent an array of tuples as a tuple of arrays, and this requires recursion:
\begin{lstlisting}[mathescape]
Typerec RecArray $\alpha$ of
| Bool => Boolarray
| Int => Intarray
| $\tau_1 \times \tau_2$ => RecArray[$\tau_1$] $\times$ RecArray[$\tau_2$]
| $\tau$ => boxedarray($\tau$)
\end{lstlisting}

It's not difficult to see how to represent and implement the non-recursive \lsti{Typecase} form.
We represent each complex branch (pairs, functions, and lists) as a function from types to types.
So the kinding rule for \lsti{Typecase} would be\footnote{The premises for booleans, strings, lists
are analogous to the ones given.}:

\begin{prooftree}
\AxiomC{$\Gamma \vdash \mu_i :: \kappa$}
\AxiomC{$\Gamma \vdash \mu :: *$}
\noLine
\BinaryInfC{$\Gamma \vdash \mu_{\times} :: * \rightarrow * \rightarrow \kappa$}
\noLine
\UnaryInfC{$\Gamma \vdash \mu_{\rightarrow} :: * \rightarrow * \rightarrow \kappa$}
\UnaryInfC{$\Gamma \vdash$ Typecase $\mu$ of $(\mu_i|\mu_{\rightarrow}|\mu_{\times}) :: \kappa$}
\end{prooftree}

To implement the recursive case, I felt that the most natural thing to do was to extend the language with
a recursion primitive at the type level, in the same way as I had at the term-level before. The idea was that
we could then rewrite \lsti{RecArray} as:
\begin{lstlisting}[mathescape]
rec \\RecArray ($\alpha$::*) :: * =>
Typecase $\alpha$ of
| bool => boolarray
| Int => intarray
| $\tau_1 \times \tau_2$ => RecArray[$\tau_1$] $\times$ RecArray[$\tau_2$]
| $\tau$ => Boxedarray($\tau$)
\end{lstlisting}

I was well aware that this would lead to the possibility of infinite looping in the typechecker (i.e. typechecking would be undecidable).  For instance, with the type-level \lsti{rec}, we can write the following
type:
\begin{lstlisting}[mathescape]
rec \\Loop ($\alpha$::*) :: * => Loop [$\alpha$]
\end{lstlisting}
But the recursion operator seemed like the most natural way to implement \lsti{Typerec}, and as a bonus, would make the type system Turing-complete and might therefore be worth exploration.

I also thought that it would be easy to avoid such infinite looping, as long as calls to the recursive operator
were structurally decreasing (which would be easy to check.). However, this turned out to be wrong. Pretty
much any use of \lsti{Typerec} would infinite loop with this implementation strategy.

For instance, the type of the subscripting function for these specially defined arrays is
$\forall \alpha. RecArray[\alpha] \rightarrow Int \rightarrow \alpha$. If we try to normalize $RecArray[\alpha]$
in this type, after one substitution step, we have
\begin{lstlisting}[mathescape]
Typecase $\alpha$ of
| bool => boolarray
| Int => intarray
| $\tau_1 * \tau_2$ => (rec \\RecArray ...)[$\tau_1$] *
            (rec \\RecArray ...)[$\tau_2$]
| $\tau$ => boxedarray($\tau$)
\end{lstlisting}
where the ellipses mark the definition of \lsti{RecArray} itself.

At the term level, this would be fine, but at the type level, we have to normalize underneath $\lambda$'s---remember that each branch is represented as a type operator---meaning we have to normalize
\lsti[mathescape]{(rec \\RecArray ...)[$\tau_1$]}. This is identical in form to what we were trying to normalize
in the first place, so we clearly get an infinite loop.

After discovering this issue, I decided it would be best just to get rid of the type-level
recursion primitive and implement \lsti{Typerec} as
Harper and Morrisett do. Their kinding (or ``formation'') rule is almost the same as the one given above:
\begin{prooftree}
\AxiomC{$\Gamma \vdash \mu_i :: \kappa$}
\AxiomC{$\Gamma \vdash \mu :: *$}
\noLine
\BinaryInfC{$\Gamma \vdash \mu_{\times} :: * \rightarrow * \rightarrow \kappa \rightarrow \kappa \rightarrow \kappa$}
\noLine
\UnaryInfC{$\Gamma \vdash \mu_{\rightarrow} :: * \rightarrow * \rightarrow \kappa \rightarrow \kappa \rightarrow \kappa$}
\UnaryInfC{$\Gamma \vdash$ Typerec $\mu$ of $(\mu_i|\mu_{\rightarrow}|\mu_{\times}) :: \kappa$}
\end{prooftree}
The only difference is that we added extra parameters to the $\mu_{\times}$ and $\mu_{\rightarrow}$ type operators.
These parameters represent the \textit{result of the Typerec evaluated at the given type's components}.

Then the equivalence rules are\footnote{I give two of the rules here. The others are similar}:

\begin{prooftree}
\AxiomC{}
\UnaryInfC{Typerec Int of $(\mu_i|\mu_{\times}|\mu_{\rightarrow}) \equiv \mu_i$}
\end{prooftree}

\begin{prooftree}
\AxiomC{}
\UnaryInfC{\stackanchor{Typerec $\mu_1 \rightarrow \mu_2$ of $(\mu_i|\mu_{\times}|\mu_{\rightarrow}) \equiv$}
           {$\mu_{\rightarrow}$ $\mu_1$  $\mu_2$
                            (Typerec $\mu_1$ of $(\mu_i|\mu_{\times}|\mu_{\rightarrow}$))
                            (Typerec $\mu_2$ of $(\mu_i|\mu_{\times}|\mu_{\rightarrow}$))}}
\end{prooftree}

My first attempt failed because it substituted too eagerly: the definition of the Typerec
was substituted throughout its body even before we could tell which branch to go down.
These new formation and equivalence rules fix that: we only substitute the Typerec when we know which branch
to go down, thereby making progress and avoiding an infinite loop.

This is the internal representation of the Typerec. The only remaining issue is to translate source code
into this form. Specifically, for the complex-type branches, we have to find all recursive Typerec calls,
and replace them with some fresh variable which we add as a parameter.

Consider the product branch of the RecArray \lsti{Typerec} from before:
\begin{lstlisting}[mathescape]
Typerec RecArray $\alpha$ of
...
| $\tau_1 \times \tau_2$ => RecArray[$\tau_1$] * RecArray[$\tau_2$]
\end{lstlisting}
What we do here is find the two recursive calls to \lsti{RecArray}, replace each of them with a fresh variables and add these variables as parameters to our type-operator, like so:
\begin{lstlisting}[mathescape]
\\$\tau_1$ $\tau_2$ $\rho_1$ $\rho_2$ => $\rho_1$ * $\rho_2$
\end{lstlisting}

To accomplish this, I wrote a sort of "generalized substitution" function, which searches for any given subexpression and replaces it with another given expression.

\subsubsection{Possible Alternative Encoding}
The so-called generalized subsitution that we need in order to implement Typerec is ugly.
And maybe the Typerec operator itself is ugly. It it most natural to represent Typerec as a type operator
that \textit{can be applied} to things, but in the representation from \cite{harper-typecase}, that view of them is merely syntactic sugar.
We can combine the best of the above two representations in a third, in which a Typerec can be applied to
an argument as in my first attempt, and in which we substitute less eagerly, thereby avoiding infinite loops,
as in the second.

In this representation, there is no type argument ``baked in'' to the construct.
That is, the following is be well-defined:

\begin{lstlisting}[mathescape]
Typerec RecArray of
| Int => Int
| Bool => Bool
...
\end{lstlisting}

And in this case, to use the Typerec we need to apply it to an argument.

The other difference from \cite{harper-typecase} is that in the complex branches, we bind the whole
operator, rather than its value at some particular argument.
So the product branch of \lsti{RecArray} in this representation would look like:

\begin{lstlisting}[mathescape]
\\$\tau_1$ $\tau_2$ RecArray => RecArray [$\rho_1$] * RecArray [$\rho_2$]
\end{lstlisting}

The formation rule would now be:

\begin{prooftree}
\AxiomC{$\Gamma \vdash \mu_i :: \kappa$}
\noLine
\UnaryInfC{$\Gamma \vdash \mu_{\times} :: * \rightarrow * \rightarrow (* \rightarrow \kappa) \rightarrow \kappa$}
\noLine
\UnaryInfC{$\Gamma \vdash \mu_{\rightarrow} :: * \rightarrow * \rightarrow (* \rightarrow \kappa) \rightarrow \kappa$}
\UnaryInfC{$\Gamma \vdash$ Typerec of $(\mu_i|\mu_{\rightarrow}|\mu_{\times}) :: * \rightarrow \kappa$}
\end{prooftree}

And the new equivalence rules would be:
\begin{prooftree}
\AxiomC{}
\UnaryInfC{(Typerec $(\mu_i|\mu_{\times}|\mu_{\rightarrow})) [Int] \equiv \mu_i$}
\end{prooftree}

\begin{prooftree}
\AxiomC{}
\UnaryInfC{(Typerec of $(\mu_i|\mu_{\times}|\mu_{\rightarrow})) [\mu_1 \rightarrow \mu_2] \equiv$
           $\mu_{\rightarrow}$ $\mu_1$  $\mu_2$
                            (Typerec $(\mu_i|\mu_{\times}|\mu_{\rightarrow}$))}
\end{prooftree}

This seems like the most natural representation for Typerec. However, I didn't implement it for lack of time.

\subsection{Void}
We're nearly there. The last step is to add a $void$ type, which is the empty type.
This was a really trivial extension to the language since nothing ever has this type!
The typing rules for the void type are:

\subsection{Encoding Typeclasses}
Returning to the to_string function from before, we want the typechecker to prevent us from calling
it on a function argument. We are now equipped with the tools to do this.
Define a type operator which maps everything to itself, except functions, which it maps to void.
\begin{lstlisting}
Typerec S :: * -> *
 | Int => Int
 | Bool => Bool
 | Str => Str
 | Void => Void
 | T1 -> T2 => Void
 | T1 * T2 => S[T1] * S[T2]
 | List t => List (S[T])
end
\end{lstlisting}

Then define to_string as:

\begin{lstlisting}[mathescape]
rec \ to_string (A::*) : S [A] -> Str =>
    typecase [t. S[t] -> str] a of
    | Int => int2string
    | Bool => bool2string
    | Str => id [Str]
    | Void => (\ (x:Void) => "")
    | T1 -> T2 => (\ (x:Void) => "")
    | T1 * T2 => (\ (p:S[T1*T2]) => 
                    "(" ++ (to_string [S[T1]] (fst p)) ++ "," ++
                           (to_string [S[T2]] (snd p)) ++ ")")
    | list t => (\ (xs:S[List T]) =>
                   fold [S[T]] [Str] 
                        (\ (v:S[T]) (acc:Str) =>
                           (to_string [S[T]] v) ++ acc) "")
    end
\end{lstlisting}

Since to_string has type $\forall \alpha. S[\alpha] \rightarrow str$, if $\alpha$ is a function type then
\lsti{to_string}$[\alpha]$ will be of type $void \rightarrow str$, and any call to it will fail to typecheck.

\appendix
\section{Appendix}

\subsection{Code}

All the code for the current version of Typsy is available at \url{https://github.com/bgrange/independent-work}.

Compile with: \texttt{ocamlbuild -use-ocamlfind typsy.d.byte}

Run with: \texttt{./typsy.d.byte <path-to-source>}

There are plenty of examples in \texttt{src/test/}. In particular, there are three examples there of advanced use of Typerec:
\begin{itemize}
\item \texttt{show.typ} is almost identical to the \texttt{to_string} example given in this report.
\item \texttt{eq.typ} defines a type-predicate for equality. Specifically, anything but function types
can be compared for equality. Then I use equality to write a generic list-lookup function.
\item \texttt{monoid.typ} defines several auxiliary type operators---including one which compares two types for
equality---and then defines the monoid type predicate. (A monoid is just a set with an associative operation and
identity.) This predicate is then used to define a generic plus function. This was only tricky

\subsection{Language Syntax Reference}

\nocite{*}
\bstctlcite{bstctl:etal, bstctl:nodash, bstctl:simpurl}
\bibliographystyle{IEEEtranS}
\bibliography{references}

\end{document}


%%% Local Variables:
%%% mode: latex
%%% TeX-master: t
%%% End:
