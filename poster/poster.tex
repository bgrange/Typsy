\documentclass[portrait,final,paperwidth=40in,paperheight=40in]{baposter}
%\documentclass[a4shrink,portrait,final]{baposter}
% Usa a4shrink for an a4 sized paper.

\tracingstats=2

\usepackage{calc}
\usepackage{graphicx}
\usepackage{amsmath}
\usepackage{amssymb}
\usepackage{relsize}
\usepackage{multirow}
%\usepackage{bm}

\usepackage{graphicx}
\usepackage{multicol}

\usepackage{pgfbaselayers}
\pgfdeclarelayer{background}
\pgfdeclarelayer{foreground}
\pgfsetlayers{background,main,foreground}

%\usepackage{times}
%\usepackage{helvet}
%\usepackage{bookman}
%\usepackage{palatino}
\usepackage{bussproofs}
\usepackage{listings}

\usepackage[scaled]{beramono}
\usepackage[T1]{fontenc}

\newcommand{\captionfont}{\footnotesize}

\selectcolormodel{cmyk}

\graphicspath{{images/}}

%%%%%%%%%%%%%%%%%%%%%%%%%%%%%%%%%%%%%%%%%%%%%%%%%%%%%%%%%%%%%%%%%%%%%%%%%%%%%%%%
% Multicol Settings
%%%%%%%%%%%%%%%%%%%%%%%%%%%%%%%%%%%%%%%%%%%%%%%%%%%%%%%%%%%%%%%%%%%%%%%%%%%%%%%%
\setlength{\columnsep}{0.7em}
\setlength{\columnseprule}{0mm}


%%%%%%%%%%%%%%%%%%%%%%%%%%%%%%%%%%%%%%%%%%%%%%%%%%%%%%%%%%%%%%%%%%%%%%%%%%%%%%%%
% Save space in lists. Use this after the opening of the list
%%%%%%%%%%%%%%%%%%%%%%%%%%%%%%%%%%%%%%%%%%%%%%%%%%%%%%%%%%%%%%%%%%%%%%%%%%%%%%%%
\newcommand{\compresslist}{%
\setlength{\itemsep}{1pt}%
\setlength{\parskip}{0pt}%
\setlength{\parsep}{0pt}%
}


%%%%%%%%%%%%%%%%%%%%%%%%%%%%%%%%%%%%%%%%%%%%%%%%%%%%%%%%%%%%%%%%%%%%%%%%%%%%%%
%%% Begin of Document
%%%%%%%%%%%%%%%%%%%%%%%%%%%%%%%%%%%%%%%%%%%%%%%%%%%%%%%%%%%%%%%%%%%%%%%%%%%%%%

\begin{document}

%%%%%%%%%%%%%%%%%%%%%%%%%%%%%%%%%%%%%%%%%%%%%%%%%%%%%%%%%%%%%%%%%%%%%%%%%%%%%%
%%% Here starts the poster
%%%---------------------------------------------------------------------------
%%% Format it to your taste with the options
%%%%%%%%%%%%%%%%%%%%%%%%%%%%%%%%%%%%%%%%%%%%%%%%%%%%%%%%%%%%%%%%%%%%%%%%%%%%%%
% Define some colors
\definecolor{silver}{cmyk}{0,0,0,0.3}
\definecolor{yellow}{cmyk}{0,0,0.9,0.0}
\definecolor{reddishyellow}{cmyk}{0,0.22,1.0,0.0}
\definecolor{black}{cmyk}{0,0,0.0,1.0}
\definecolor{darkYellow}{cmyk}{0,0,1.0,0.5}
\definecolor{darkSilver}{cmyk}{0,0,0,0.1}

\definecolor{lightyellow}{cmyk}{0,0,0.3,0.0}
\definecolor{lighteryellow}{cmyk}{0,0,0.1,0.0}
\definecolor{lighteryellow}{cmyk}{0,0,0.1,0.0}
\definecolor{lightestyellow}{cmyk}{0,0,0.05,0.0}

\lstset{language=Caml,
        basicstyle=\ttfamily\small,
        morekeywords={typecase,Typerec,Typecase}}
\newcommand{\lsti}{\lstinline}

%%
%\typeout{Poster Starts}
%\background{
%  \begin{tikzpicture}[remember picture,overlay]%
%    \draw (current page.north west)+(-2em,2em) node[anchor=north west] {\includegraphics[height=1.1\textheight]{silhouettes_background}};
%  \end{tikzpicture}%
%}

\newlength{\leftimgwidth}
\begin{poster}%
  % Poster Options
  {
  % Show grid to help with alignment
  grid=false,
  % Column spacing
  colspacing=1em,
  % Color style
  bgColorOne=lighteryellow,
  bgColorTwo=lightestyellow,
  borderColor=reddishyellow,
  headerColorOne=yellow,
  headerColorTwo=reddishyellow,
  headerFontColor=black,
  boxColorOne=lightyellow,
  boxColorTwo=lighteryellow,
  % Format of textbox
  textborder=roundedleft,
%  textborder=rectangle,
  % Format of text header
  eyecatcher=false,
  headerborder=open,
  headerheight=0.08\textheight,
  headershape=roundedright,
  headershade=plain,
  boxshade=plain,
%  background=shade-tb,
  background=plain,
  linewidth=2pt
  }
  {}
    % Title
  {Encoding Typeclasses in System F$_{\omega}$}
  % Authors
  {Ben Grange, with Prof. David Walker}
  % University logo
  {% The makebox allows the title to flow into the logo, this is a hack because of the L shaped logo.
    \makebox[8em][r]{%
      \begin{minipage}{16em}
        \hfill
        \includegraphics[height=2em]{pton_logo}
      \end{minipage}
    }
  }

  \tikzstyle{light shaded}=[top color=baposterBGtwo!30!white,bottom color=baposterBGone!30!white,shading=axis,shading angle=30]

  % Width of left inset image
     \setlength{\leftimgwidth}{0.78em+8.0em}

%%%%%%%%%%%%%%%%%%%%%%%%%%%%%%%%%%%%%%%%%%%%%%%%%%%%%%%%%%%%%%%%%%%%%%%%%%%%%%
%%% Now define the boxes that make up the poster
%%%---------------------------------------------------------------------------
%%% Each box has a name and can be placed absolutely or relatively.
%%% The only inconvenience is that you can only specify a relative position 
%%% towards an already declared box. So if you have a box attached to the 
%%% bottom, one to the top and a third one which should be in between, you 
%%% have to specify the top and bottom boxes before you specify the middle 
%%% box.
%%%%%%%%%%%%%%%%%%%%%%%%%%%%%%%%%%%%%%%%%%%%%%%%%%%%%%%%%%%%%%%%%%%%%%%%%%%%%%
    %
    % A coloured circle useful as a bullet with an adjustably strong filling
    \newcommand{\colouredcircle}[1]{%
      \tikz{\useasboundingbox (-0.2em,-0.32em) rectangle(0.2em,0.32em); \draw[draw=black,fill=baposterBGone!80!black!#1!white,line width=0.03em] (0,0) circle(0.18em);}}

%%%%%%%%%%%%%%%%%%%%%%%%%%%%%%%%%%%%%%%%%%%%%%%%%%%%%%%%%%%%%%%%%%%%%%%%%%%%%%
  \begin{posterbox}[name=overview,row=0,column=0]{Overview}
%%%%%%%%%%%%%%%%%%%%%%%%%%%%%%%%%%%%%%%%%%%%%%%%%%%%%%%%%%%%%%%%%%%%%%%%%%%%%%
   Typeclasses are a powerful construct in Haskell for defining functions
   differently on the type of an argument (ad-hoc polymorphism).
   
   In GHC, typeclasses are implemented with ``dictionary-passing''.
   I investigated an alternative approach---encoding typeclasses in terms of the
   more the primitive typecase and Typerec constructs---by designing a small
   language from scratch. 
  \vspace{0.3em}

 \end{posterbox}

%%%%%%%%%%%%%%%%%%%%%%%%%%%%%%%%%%%%%%%%%%%%%%%%%%%%%%%%%%%%%%%%%%%%%%%%%%%%%%
  \begin{posterbox}[name=untyped,below=overview]{Untyped Lambda Calculus}
%%%%%%%%%%%%%%%%%%%%%%%%%%%%%%%%%%%%%%%%%%%%%%%%%%%%%%%%%%%%%%%%%%%%%%%%%%%%%%
 
  The untyped lambda calculus is the basis for everything. Terms are either
  variables, $\lambda$ abstractions, or function applications.
  The system can be described by these rules:

  \begin{prooftree}
  \AxiomC{$e_1 \rightarrow e_1'$}
  \UnaryInfC{$e_1 e_2 \rightarrow e_1'e_2$}
  \end{prooftree}

  \begin{prooftree}
  \AxiomC{value $v$}
  \AxiomC{$e_2 \rightarrow e_2'$}
  \BinaryInfC{$v \, e_2 \rightarrow v \, e_2'$}
  \end{prooftree}

  \begin{prooftree}
  \AxiomC{value $v$}
  \UnaryInfC{$(\lambda x.e_1)v \rightarrow [x \mapsto v]e_1$}
  \end{prooftree}   

  An example computation:

  \begin{lstlisting}[mathescape]

     ($\lambda$xy.y)($\lambda$xy.x)($\lambda$xy.xy) $\rightarrow$ ($\lambda$y.y)($\lambda$xy.xy)
                            $\rightarrow$ $\lambda$xy.xy

  \end{lstlisting}

  We can do any computation in this system, but it's highly inconvenient.

  \end{posterbox}

%%%%%%%%%%%%%%%%%%%%%%%%%%%%%%%%%%%%%%%%%%%%%%%%%%%%%%%%%%%%%%%%%%%%%%%%%%%%%%
  \begin{posterbox}[name=simple,below=untyped]{Simply Typed Lambda Calculus}
%%%%%%%%%%%%%%%%%%%%%%%%%%%%%%%%%%%%%%%%%%%%%%%%%%%%%%%%%%%%%%%%%%%%%%%%%%%%%%
    
  The language was enriched with the primitive types \lsti{Int}, \lsti{Bool}, \lsti{Str},
  \lsti{List}, and \lsti{Void}. (\lsti{Void} will become important later).

  Primitive types allow us to compute more naturally, but they create the possibility
  of stuck (i.e. crashed) programs. The typechecker to statically guarantee type safety.

  The fundamental typing rules are:


  \begin{prooftree}
  \AxiomC{$x \in \Gamma$}
  \UnaryInfC{$\Gamma \vdash x : T$}
  \end{prooftree}

  \begin{prooftree}
  \AxiomC{$\Gamma \cup \{x:T_1\} \vdash e:T_2$}
  \UnaryInfC{$\Gamma \vdash (\lambda x:T_1.e) : T_1 \rightarrow T_2$}
  \end{prooftree}

  \begin{prooftree}
  \AxiomC{$\Gamma \vdash e_1 : T_1 \rightarrow T_2$}
  \AxiomC{$\Gamma \vdash e_2 : T_1$}
  \BinaryInfC{$\Gamma \vdash e_1 e_2 : T_2$}
  \end{prooftree}

  \vspace{0.3em}
  \end{posterbox}

%%%%%%%%%%%%%%%%%%%%%%%%%%%%%%%%%%%%%%%%%%%%%%%%%%%%%%%%%%%%%%%%%%%%%%%%%%%%%%
  \begin{posterbox}[name=poly,column=1]{System F}
%%%%%%%%%%%%%%%%%%%%%%%%%%%%%%%%%%%%%%%%%%%%%%%%%%%%%%%%%%%%%%%%%%%%%%%%%%%%%%
  
  In STLC, we have no polymorphism.

  We can write a map function, but it must be specific to a pair of types:

  \begin{lstlisting}
    let rec map (f:Int -> Str)
                (l:List Int) : List Str =
    ...
  \end{lstlisting}

  To map over a list of any other type, we have to define a new function:

  \begin{center}
    \line(1,0){250}
  \end{center}

  To implement parametric polymorphism, I added type functions, from types
  to values. Now, map can be written as one such function:

  \begin{lstlisting}
    let rec map (A::*) (B::*)
                (f:A -> B) (l:List A) : List B =
    ...
  \end{lstlisting}

  where \lsti{(A::*)} means that \lsti{A} is a type parameter.

  We can then call \lsti{map} like so:

  \begin{lstlisting}[mathescape]
    map [Int] [Int] ($\lambda$(x:Int).x+1) [1,2,3]
  \end{lstlisting}

  \end{posterbox}

%%%%%%%%%%%%%%%%%%%%%%%%%%%%%%%%%%%%%%%%%%%%%%%%%%%%%%%%%%%%%%%%%%%%%%%%%%%%%%
  \begin{posterbox}[name=tyop,column=1,below=poly]{System F_$\omega$}
%%%%%%%%%%%%%%%%%%%%%%%%%%%%%%%%%%%%%%%%%%%%%%%%%%%%%%%%%%%%%%%%%%%%%%%%%%%%%% 
    
  For the purpose of implementing typeclasses, I added type operators,
  which are functions from types to types.

  For instance, to create an association-list type,
  we could use the type operator:

  \begin{lstlisting}[mathescape]
    $\lambda$(X::*)(Y::*).List (X $\times$ Y)
  \end{lstlisting}

  Base types have kind $\ast$, while
  this type operator has ``higher kind'' $\ast \rightarrow \ast$.

  \end{posterbox}

%%%%%%%%%%%%%%%%%%%%%%%%%%%%%%%%%%%%%%%%%%%%%%%%%%%%%%%%%%%%%%%%%%%%%%%%%%%%%%
  \begin{posterbox}[name=tcase,column=1,below=tyop]{typecase/Typerec}
%%%%%%%%%%%%%%%%%%%%%%%%%%%%%%%%%%%%%%%%%%%%%%%%%%%%%%%%%%%%%%%%%%%%%%%%%%%%%% 
    
  The \lsti{typecase} and \lsti{Typerec} constructs allow us to do
  runtime type analysis in a typesafe manner.

  So, to define a generic equality function:

  \begin{lstlisting}[mathescape]
  let eq (A::*) : A $\rightarrow$ A $\rightarrow$ Bool =
  typecase A of
  | Int $\Rightarrow$ inteq
  | T$_1$ $\rightarrow$ T$_2$ $\Rightarrow$ $\lambda$ f1 f2.false
  | T$_1$ $\ast$ T$_2$ $\Rightarrow$ $\lambda$ (x1,y1) (x2,y2).(eq [T$_1$] x1 x2) and
                                 (eq [T$_2$] y1 y2)

  \end{lstlisting}

  \lsti{Typerec} allows us to do the same thing at the type-level.
  So to define a different representation of arrays depending on the
  array element type:

  \begin{lstlisting}[mathescape]
    Typerec RecArray $\alpha$ of
    | Bool => Boolarray
    | Int => Intarray
    | $\tau_1 \times \tau_2$ => RecArray[$\tau_1$] $\times$ RecArray[$\tau_2$]
    | $\tau$ => boxedarray($\tau$)
  \end{lstlisting}

  Note that we can do recursion on the type, but the typesystem remains
  decideable.
  \end{posterbox}

%%%%%%%%%%%%%%%%%%%%%%%%%%%%%%%%%%%%%%%%%%%%%%%%%%%%%%%%%%%%%%%%%%%%%%%%%%%%%%
  \begin{posterbox}[name=tclass,column=2]{Typeclasses in Haskell}
%%%%%%%%%%%%%%%%%%%%%%%%%%%%%%%%%%%%%%%%%%%%%%%%%%%%%%%%%%%%%%%%%%%%%%%%%%%%%% 
    
  Typeclasses allow us to overload functions easily.
  We define the function we want to overload:

  \begin{lstlisting}[language=Haskell]
    class Eq a where
    (==) :: a -> a -> Bool
  \end{lstlisting}

  Then we define it differently on different types:
  \begin{lstlisting}[language=Haskell]
    instance Eq a => Eq [a] where
    xs == ys = all (\ (x,y) -> x == y) (zip xs ys)
  \end{lstlisting}

  \begin{lstlisting}[language=Haskell,mathescape]
    instance Eq a => Eq b => Eq (a $\times$ b) where
    (x1,y1) == (x2,y2) = x1 == x2 and y1 == y2
  \end{lstlisting}

   If we define a new type that isn't an instance of \lsti[language=Haskell]{Eq}
  and try to test it for equality, we'll get a type error.
  So there are two critical features here that we want to emulate:
  overloading (ad-hoc polymorphism), and partiality.

  \end{posterbox}

%%%%%%%%%%%%%%%%%%%%%%%%%%%%%%%%%%%%%%%%%%%%%%%%%%%%%%%%%%%%%%%%%%%%%%%%%%%%%%
  \begin{posterbox}[name=tclassenc,column=2,below=tclass]{Encoding Typeclasses}
%%%%%%%%%%%%%%%%%%%%%%%%%%%%%%%%%%%%%%%%%%%%%%%%%%%%%%%%%%%%%%%%%%%%%%%%%%%%%% 
    The \lsti{typecase} construct gives us overloading really easily.

    We can then use \lsti{Void} to achieve partiality.

    Say we want to define an equality function for all types but function types
    (and types containing function types, e.g. pairs of functions)

    Define a \lsti{Typerec} which maps every type to itself, except functions
    which go to void.

    \begin{lstlisting}[mathescape]
      Typerec Eq [X] :: * of
      | Int $\Rightarrow$ Int
      | Bool $\Rightarrow$ Bool
      | Str $\Rightarrow$ Str
      | Void $\Rightarrow$ Void
      | S $\rightarrow$ T $\Rightarrow$ Void
      | S $\times$ T $\Rightarrow$ (Eq [S]) $\times$ (Eq [T])
      | List S $\Rightarrow$ List (Eq [S])
      end
    \end{lstlisting}

    Then we can define our equality function like so:

    \begin{lstlisting}[mathescape]
    let rec eq (A::*) : Eq [A] $\rightarrow$ Eq [A] $\rightarrow$ Bool =
      typecase A of
      | Int $\Rightarrow$ inteq
      | Bool $\Rightarrow$ booleq
      | Str $\Rightarrow$ streq
      | Void $\Rightarrow$ $\lambda$ v1 v2. false
      | S $\rightarrow$ T $\Rightarrow$ $\lambda$ v1 v2. false
      | S $\times$ T $\Rightarrow$
             $\lambda$ p1 p2. eq [S] (fst p1) (fst p2) and
                      eq [T] (snd p1) (snd p2)
      | List S $\Rightarrow$ eq_list [Eq[S]] (eq [S])
      end
    \end{lstlisting}
    
  Typeclasses as written above could easily be translated into this form, although I did not do so.

  \end{posterbox}


\end{poster}

\end{document}

%%% Local Variables:
%%% mode: latex
%%% TeX-master: t
%%% End:
